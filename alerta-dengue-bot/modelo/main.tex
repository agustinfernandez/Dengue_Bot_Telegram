\documentclass[a4paper,10pt]{article}
\usepackage[utf8]{inputenc}
\input{../auxiliar/encabezado.tex}
\input{../auxiliar/tikzlibrarybayesnet.code.tex}

 
%opening
\title{Prevenci\'on temprana de vectores de dengue}
\author[1,2]{et al}

\affil[1]{\small Fundaci\'on Soberan\'ia Sanitaria}
\affil[2]{\small Universidad de Buenos Aires. Facultad de Ciencias Exactas y Naturales. Departamento de Computaci\'on. Buenos Aires, Argentina}
\affil[]{Correspondencia: \url{gustavolandfried@gmail.com}}


\begin{document}

\maketitle

\newpage

\section{Modelo}

\begin{description}
 \item[$C$)] Criaderos. Densidad en metros cuadrados.
 \item[$V$)] Vectores. Densidad
 \item[$L$)] Larvas. Densidad
 \item[$P$)] Poblaci\'on. Densidad
 \item[$R$)] Reporte de le promotore de salud.
 \item[$S$)] Sensibilidad de le promotore de salud.
\end{description}

\begin{figure}[H]
\centering
\tikz{ 
    
    
    \node[latent] (c) {$C$} ; %
    \node[latent, right= of c] (l) {$L$} ; %
    \node[latent, below= of l] (v_t0) {$V_{t}$} ; %
    \node[latent, right= of l] (v_t1) {$V_{t+1}$} ; %
    \node[latent, right= of v_t1] (d) {$D$} ; %

    \node[latent, left= of c] (s) {$S$} ; %
    \node[latent, below= of s, xshift=0.75cm, fill=black!15] (r) {$R$} ; %
    
    \edge {c,v_t0} {l};
    \edge {l} {v_t1};
    \edge {v_t1} {d};
    
    \edge {s,c} {r}
    
%         \node[det, fill=black!10] (r1) {$r_1$} ; %
%         \node[det, fill=black!10, xshift=3cm] (r2) {$r_2$} ; %
%         
%         \node[latent, above=of r1] (d1) {$d_1$} ; %
%         \node[latent, above=of r2] (d2) {$d_2$} ; %
%         
%         \node[latent, above=of d1, xshift=-1.5cm] (t1) {$t_1$} ; %
%         \node[latent, above=of d1,xshift=1.5cm] (t2) {$t_2$} ; %
%         \node[latent, above=of d2, xshift=1.5cm] (t3) {$t_3$} ; %
%         
%          
%         
%         \node[latent, above=of t1, xshift=-0.8cm] (p1) {$p_1$} ; %
%         \node[latent, above=of t1, xshift=0.8cm] (p2) {$p_2$} ; %
% 
%         \node[latent, above=of t2] (p3) {$p_3$} ; %
%         
%         \node[latent, above=of t3, xshift=-0.8cm] (p4) {$p_4$} ; %
%         \node[latent, above=of t3, xshift=0.8cm] (p5) {$p_5$} ; %
% 
%         \node[latent, above=of p1] (s1) {$s_1$} ;
% 
%         \node[latent, above=of p2] (s2) {$s_2$} ;
%         \node[latent, above=of p3] (s3) {$s_3$} ;
%         \node[latent, above=of p4] (s4) {$s_4$} ;
%         \node[latent, above=of p5] (s5) {$s_5$} ;
%         
%         \edge {s1} {p1};
%         \edge {s2} {p2};
%         \edge {s3} {p3};
%         \edge {s4} {p4};
%         \edge {s5} {p5};
%         
%         \edge {p1,p2} {t1};
%         \edge {p3} {t2};
%         \edge {p4,p5} {t3};
%         
%         \edge {t1} {d1};
%         \edge {t2} {d1,d2};
%         \edge {t3} {d2};
%         
%         \edge {d1} {r1};
%         \edge {d2} {r2};
%         \node[const, xshift=7.3cm] (result-dist) {$r_j = d_j > 0$} ; %
% 	\node[const, above=of result-dist,yshift=1.2cm] (d-dist) {$d_j = t_{j} - t_{j+1}$};  %
% 	\node[const, above=of d-dist,yshift=1.15cm] (t-dist) {$t_j = \sum_{i\in A_j} p_i $} ; %
% 	\node[const, above=of t-dist,yshift=1.3cm] (p-dist) {$p_i \sim N(s_i,\beta^2)$} ; %
% 	\node[const, above=of p-dist,yshift=1.3cm] (s-dist) {$s_i \sim N(\mu_i,\sigma_i^2)$} ; %
% 	
% 	
%           \node[invisible, right=of result-dist, xshift=1.5cm] (inv) {};
} 
\caption{Modelo causal}
\end{figure}






\end{document}
